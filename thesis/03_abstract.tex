\section*{Abstract}
This document outlines a Master thesis topic. It summaries its relevance, background and related research, describes the data used, proposes methods, and exemplifies the research idea on a small data subsample.
The proposed research aims at implementing prerequisites for the employment of local energy markets on a distributed ledger technology. Such local energy markets have been described and simulated on a Blockchain in previous literature \cite[e.g.,][]{Mengelkamp:2018a}). However, the focus of previous literature has been on the problem of programming such a decentralized market platform in form of a Smart Contract . Fundamental prerequisites for this Smart Contract, such as reliable Smart Meter forecasts of households’ energy consumption and production, have been assumed as given. Thus, how forecasts with a reasonably small error can be computed, is neglected in this literature. This task is particularly challenging under the constraints imposed by the technical implementation of a market mechanism in a Smart Contract on a Blockchain.
Therefore, the proposed study aims to evaluate the possibility of providing such forecasts with existing forecasting methods and realistically available Smart Meter data. An extensive literature review of studies that use high-resolution Smart Meter data to predict individual household energy consumption was conducted. Based on these studies the following forecast-ing techniques seemed most promising and applicable to the task at hand: long short-term memory recurring neural networks (LSTM RNN), pooling-based deep recurring neural net-works (PD RNN), sparse autoregressive LASSO, and distinct wavelet transform (DWT) with Kernel-Wavelet-Functional prediction method (KWF). The best performing prediction technique in terms of four error measures (i.e., MAE, MAPE, NRSME, and MASE) will then be used to assess the effect of forecasting errors on the market mechanism described in \citet{Mengelkamp:2018a}.

%%%%%%%%%%%%%%%%%%%%%%%%%%%%%%%%%%%%%%%%%%%%%%%%%%%%%%%%%%%%%%%%%
This is the template for a thesis at the Chair of Econometrics of
Humboldt--Universit\"at zu Berlin. A popular approach to write a
thesis or a paper is the IMRAD method (Introduction, Methods,
Results and Discussion). This approach is not mandatory! You can
find more information about formal requirements in the booklet
`Hinweise zur Gestaltung der \"au\ss eren Form von Diplomarbeiten' which is available in the office of studies.

The abstract should not be longer than a paragraph of around 10 to 15 lines (or about 150 words). The abstract should contain a
concise description of the econometric/economic problem you
analyse and of your results. This allows the busy reader to obtain quickly a clear idea of the thesis content.
