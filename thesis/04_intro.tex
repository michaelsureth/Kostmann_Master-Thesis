
\section{Introduction}\label{Sec:Intro}



%%%%%%%%%%%%%%%%%%%%%%
%%%   Motivation   %%%
%%%%%%%%%%%%%%%%%%%%%%

\subsection{Motivation}\label{Sec:Intro;Subsec:Motivation}

The increasingly wide-spread installation of renewable energy generators currently transforms the German energy landscape substantially \citep{Bayer:2018}. Already in 2017, more than 1.6 million photovoltaic micro-generation units were installed in Germany, according to the Bundesverband Solarwirtschaft \citep{BSW-Solar:2018}. This increasing amount of distributed renewable energy resources combined with a more volatile energy consumption of households –- e.g., due to uncontrolled electric vehicle charging that can increase peak consumption \citep{Fitzgerald:2016,Floch:2017} -– presents a serious challenge for grid operators. As in any electricity grid energy production and consumption have to be balanced at all times \citep{Weron:2006}, the increasingly volatile and hard to predict energy consumption and production in low voltage grids require new technological solution to manage grid load and energy distribution.

Fortunately, the technological advancement that lead to the increasing complexity in the energy landscape, also opens up new opportunities to increase the efficiency and reliability of distributed renewable energy production and distribution. As the amount of renewable energy production, that is fed into low voltage grids, has been increasing over the last years \citep{Bayer:2018}, it seems reasonable to shift part of the grid management to lower grid levels. While industry and research already established a comprehensive set of grid management solutions as well as sophisticated consumption and production forecasting techniques for highly aggregated levels, there is still little research on the same topics at lower aggregation levels, such as neighbourhoods or even individual households \citep{Meer:2018}.

One rather recent technological advancement that has the potential to increase the level of energy distribution efficiency on low aggregation level is the implementation of local energy markets on a distributed ledger technology such as blockchain. Blockchain has been called an invention similarly revolutionary and paradigm shifting as the Internet \citep{Swan:2015}. While much of the hype around blockchain still has to stand the test against reality, the technology undeniably has the potential to enable new technological solutions. It is not for no reason that more than 20~\% of 70 surveyed German energy executives believe blockchain will be “a game changer for the energy industry” and further 60~\% believe further dispersion of blockchain technology is probable \citep{Burger:2016}. A use case that has been getting special attention due to the media-effective inauguration of the Brooklyn Microgrid \citep{newscientist:2016} are blockchain-based local energy markets.

Local energy markets (LEM) enable localized interconnected energy consumers, producers, and prosumers to trade locally produced energy\footnote{In this thesis, the terms energy and electricity are used interchangeably as is common in related literature. However, to be precise, the term energy comprises electricity and heat and a local energy market does not necessarily has to be constrained to the trading of electricity. Still, all further mentions of the term ``energy'' in this work refer to electricity.} on a market platform with a specific pricing mechanism \citep{Mengelkamp:2018a}. A common pricing mechanism used for this purpose are discrete double auctions\citep{Lamparter:2010, Buchmann:2013, Block:2008}. Major advantages of such local energy markets are (near) real-time pricing and balancing of energy production and consumption in local grids (ibid.). Blockchain-based LEM utilize a blockchain as underlying information and communication technology and a smart contract to match supply and demand and settle transactions (ibid.).

However, the product traded on energy markets has some peculiarities compared to other goods. First, energy grids always have to be balanced, i.e. energy demand always has to be matched by energy supply \citep{Weron:2006}. Secondly, as energy is difficult to store, produced energy is fed into the grid mostly instantaneously and continuously and cannot be exchanged in batches of a specific amount at a single point in time. Traditionally, this means that the aggregated energy demand for a geographic area and a specific period of time has to be anticipated and according to this future demand energy is bought and sold. The actual electricity is then produced continuously matching the current demand. This setting is the reasons for today’s existing energy landscape, where utilities and large-scale energy producers and consumers are the only agents involved in electricity markets \citep{Weron:2006}. They trade energy according to the aggregated demand of many consumers. This aggregation makes forecasting future energy demand with relatively small errors \citep{Meer:2018, Wang:2018} and thereby efficient trading possible. Household-level consumers or prosumers, however, do not actively trade but pay their consumption or are reimbursed for their infeed of energy into the grid according to preset tariffs. 

In LEMs, on the contrary, households are the participating market agents that typically submit offers in an auction design. Due to the non-storability of the traded good, the participating households need to forecast their energy demand, respectively supply, to be able to submit a buy or sell offer to the market. Therefore, accurate forecasts are a necessary precondition for such market designs. However, even though forecasting is substantially harder for single households compared to higher aggregation levels \citep{Wang:2018}, in existing research on (blockchain-based) LEM, it is frequently assumed such accurate forecasts are readily available \citep{Rosen:2013, Mengelkamp2018c, Lamparter:2010, Buchmann:2013, Mengelkamp:2018a}. This assumption may not be correct and given the substantial uncertainty in individual households' energy consumption or production, prediction errors may have a significant impact on market outcomes.

Therefore, the present research aims to evaluate the possibility of providing such forecasts with existing forecasting methods and realistically available smart meter data with reasonable accuracy. Moreover, it aims to quantify the effect of prediction errors on market outcomes. Specifically for local energy markets, this has not been done in previous literature. However, for the future advancement of the field it seems imperative that the precondition of accurate forecasting for local energy markets is sufficiently assessed such that -- if necessary -- the assumption of readily available accurate forecasts can be adjusted in future work.

%%%%%%%%%%%%%%%%%%%%%%%%%%%%
%%%   Related research   %%%
%%%%%%%%%%%%%%%%%%%%%%%%%%%%

\subsection{Related research}\label{Sec:Intro;Subsec:Related}
The present work's topic of concern touches upon three fields of research. The first superordinate topic is local energy markets, their market structure, market mechanism, and market outcomes as well as possible advantages and disadvantages. The second superordinate topic is distributed ledger technology and as such blockchain and smart contracts as well as their use cases for different fields. The third topic is energy forecasting, which encompasses energy consumption forecasting and energy production forecasting. Especially the latter has attracted rising attention in the light of increasing adoption of renewable energy resources. All this comes together in blockchain-based local energy markets as, for example, implemented in the Brooklyn Microgrid and simulated in the work by \citet{Mengelkamp:2018a}. For such blockchain-based LEM, a necessary prerequisite is the successful forecast of household-level energy consumption/production based on smart meter recordings. Without this, trading on a market mechanisms as described in, e.g., \citet{Block:2008} or \citet{Buchmann:2013}, and implemented in a smart contract by \citet{Mengelkamp:2018a} will not possible.


%%%%%%%%%%%
\subsubsection{Local energy markets}

Early work by \citet{Alibhai:2004} describe auctions as a coordination mechanism for microgrids. In their setting, energy producers within the market bid on requested electricity amounts by consumers. They compare different auction designs and come to the conclusion that Dutch auctions are the most preferable for their application.

\citet{Block:2008} are one of the few studies specifically referring to electricity and heat in their design of a local energy market. They propose a combinatorial double auction that sets a uniform equilibrium price in discrete time intervals and analytically develop a open book call market utilizing an arbitrage agent and spinning reserves to stabilize the local market.

Other early work focused on microgrid in island mode (i.e., autarchic and disconnected from a superordinate grid, such as on remote islands) and the setting of a uniform market clearing price in single-sided and double-sided auction designs \citep{Sinha:2008}. While this is especially interesting for developing regions, more recent work focused on use cases in very developed and highly technologized energy grid systems. This is mainly driven by the wide spread adoption of smart meters and internet-connected home appliances.

\citet{Lamparter:2010} introduce a fully flexible and modular market platform that coordinates market agents through a mechanism that incentivizes truthful policy revelation (i.e., bidding behaviour). Their developed software platform D'ACCORD uses a double auction (Vickrey-Clark-Groves auction) that allows for divisible bids to achieve highly efficient market outcomes.

A market mechanism that is applied in a real world project is developed by \citet{Ilic:2012}. Again, a double auction with discrete time slots is used to achieve a high market efficiency with price behaviour that is expected from standard economic theory. Using almost the same market mechanism and simulation design, \citet{Buchmann:2013} focus on a new aspect that will become more important in future applications of LEM. They tackle the problem of lacking privacy that is present in any local energy market which conforms with current German energy trading regulation. Using common anonymization methods they show that protecting the privacy of trading agents comes only at moderate cost in terms of higher prices and lower market efficiency.

\citet{Rosen:2013} bring up the important aspect of easy understandability that is needed for successful implementations of local energy markets and focus on establishing a market mechanism that is appropriate also in settings with few market participants. This is especially in early stages of LEMs an often neglected but all the more important aspect.

Finally, very recently, a series of several studies researches the usefulness of automated trading agents. Comparing zero intelligence and intelligent trading agents in two different market scenarios, \citet{Mengelkamp:2017:Trading} establish the general usefulness of automated trading agents to achieve efficient market outcomes. This work is extended in \citet{Mengelkamp:2018:Clustering} that assesses the feasibility of representing household preferences with intelligent trading agents. \citet{Mengelkamp2018c} then improve the performance of the intelligent trading agents employing reinforcement learning in a short-term merit order market mechanism.



%%%%%%%%%%%
\subsubsection{Blockchain and smart contracts}

This recently renewed research interest in LEM appeared more or less simultaneously to the exploding interest in the revolutionary distributed ledger technology, most notably blockchain \citep{Swan:2015}. As the focus of this work does not require a detailed understanding of distributed ledger technology, the in-depth explanation of its functioning is not required here. In short, blockchain can be described as a distributed record keeper (a "ledger", i.e., a database) that contains transaction between participating agents, called nodes \citep{Burger:2016}. Distributed here means that a copy of the same database (or a shortened version) is stored on each node. Summarizing \citet{Tapscott:2016}, each transaction that is executed on a blockchain is added to a so-called block. Each block contains a preset number of transactions and has to be verified by a majority of participating nodes to be added ("chained") to the distributed ledger. This addition is secured through cryptography. That means, any party trying to manipulate previous transactions would have to change all subsequent blocks of the blockchain on a majority of the participating nodes. As this would be computationally extremely demanding, it is extremely unlikely, giving blockchain its lauded characteristics of unalterability and secureness \citep{Burger:2016}.

For this thesis, a variant of the original blockchain technology is relevant: Ethereum is an open source platform built on blockchain technology. It can serve as infrastructure for any kind of blockchain-based application, cryptocurrency, protocol, and the like. On the Ethereum blockchain, any kind of programmable task can be implemented in an immutable, transparent, and distributed way. Due to the open source nature of Ethereum it is possible to “clone” the public Ethereum blockchain onto a private machine and use it as a private blockchain for simulation, testing or closed commercial applications. \citep{Ethereum:2018doc, Swan:2015}.

Another closely related term often mentioned in combination with blockchain technology is "smart contract". The concept and term smart contract dates back to Nick Szabo, who defined a smart contract as “a computerized transaction protocol that executes the terms of a contract” \citep{szabo:1994}. Simplified, a smart contract can be described as software or hardware that represents contractual clauses and can automatically registers and initialize the fulfilment of its terms while penalizing a contracting party in case of any violation of the contract \citep{Szabo:1997}. For example, this contractual clauses can also represent a market mechanism that is used to trade energy in a local market. As such it is implemented by \citet{Mengelkamp:2018a}.


%%%%%%%%%%%
\subsubsection{Local energy markets and blockchain technology}

While substantial work regarding LEM in general has been done, there are only few examples of blockchain-based LEM designs in the existing literature. \citet{Mengelkamp:2018b} derive seven principles for microgrid energy markets and evaluate the Brooklyn Microgrid according to those principles. According to the authors knowledge, they are the only ones providing a theoretical framework for the design of blockchain-based LEM and their work may serve as the basis for the future research and implementation of such energy markets. With a more practical focus, \citet{Mengelkamp:2018a} implemented and simulated a local energy market on a private Ethereum-blockchain  that enables participants to trade local energy production on a decentralized market platform with no need for a central authority. \citet{Münsing:2017} similarly elaborate a peer-to-peer energy market concept on a blockchain but focus on operational grid constraints and a fair payment rendering. In doing so, they present a decentralized optimal power flow model suitable for implementation on a blockchain.

Outside of academia there are several undertakings to put blockchain-based energy trading into practice. Prominent examples of such projects are Grid Singularity (\url{www.gridsingularity.com}) in Austria, Powerpeers (\url{www.powerpeers.nl}) in the Netherlands, Power Ledger (\url{www.powerledger.io}) in Australia, and LO3 Energy (\url{www.lo3energy.com}) in the US.



%%%%%%%%%%%
\subsubsection{Load forecasting of individual households}

As mentioned before, the simulations and concepts described in the research above rely on smart meters capable of forecasting the expected energy consumption or production of a household for the next trading interval. This forecasting task is not trivial due to the extremely high volatility of individual private energy consumption \citep{Wang:2018}. Nevertheless, there are several studies trying to forecast different time horizons of smart meter time series.

\citet{Arora:2016} compute probability density estimates for the electricity consumption recorded by individual smart meters in halfhourly intervals from 1000 households and SMEs in Ireland over the course of one year. They employ unconditional and conditional kernel density estimators with a decay parameter to generate point and density forecasts for electricity consumption from 30 minutes to one week ahead.

\citet{Kong:2018} use a long short-term memory deep learning framework to make one time-step ahead forecasts on the AMPds dataset containing half-hourly recordings of energy and appliance usage measurements of a single household in Canada. They show that the prediction accuracy can be improved substantially by including appliance measurement data.

Contrary to this machine learning approach, \citet{Li:2017} use statistical methods to make one time-step ahead forecasts with a sparse autoregressive LASSO model. Using a dataset of 150 consumers from PG\&E with hourly energy consumption recordings for one year, their model captures sparsity in the household’s historical data via LASSO to make a prediction for one household. This prediction is further improved with the historical consumption data of one additional household. This household is identified with the help of a covariance statistic test to identify one other household's data that has the best predictive leverage to improve the original forecast.

On the same dataset as \citet{Arora:2016}, \citet{Shi:2017} use a pooling-based deep recurring neural network to make point forecasts of future consumption and achieve substantial mean absolute percentage error (MAPE) reductions compared to ARIMA, recurring neural network, support vector machine, and deep recurring neural network approaches.

Even though focusing on the forecast of aggregated energy consumption, the work of \citet{Zufferey:2017} shows promising results for forecasting smart meter time series with time delay neural networks (TDNN) using mostly historical features of the time series itself. They use a huge dataset of 40.000 small consumers and 400 photovoltaic power generators in Basel, Switzerland with 15-minute interval recordings of energy consumption and production for one year.

A comprehensive overview on the state of the art of smart meter data analytics is provided by \citet{Wang:2018}. The authors do not only focus on studies researching load forecasting but also provide comprehensive insights into studies regarding smart meter data clustering, preprocessing, load analysis and more. Furthermore, they provide a summary of publicly available smart meter datasets and open research topics.

Notably, there is a lack of standard regarding which forecasting error measures are reported and what benchmark models are used in smart meter data forecasting studies. This is also pointed out by \citet{Meer:2018} in their review paper on probabilistic consumption and production forecasting. Due to this, different forecasting techniques employed in studies using different datasets with partly differing objectives are not directly comparable. 

% Which forecasting techniques should be used in the research proposed here is therefore not directly inferable from the success in applying different forecasting techniques in previous studies.



%%%%%%%%%%%%%%%%%%%%%%%%%%%%
%%%   Present research   %%%
%%%%%%%%%%%%%%%%%%%%%%%%%%%%
\subsection{Present research}\label{Sec:Intro;Subsec:Present}

The objective of the Master thesis is to investigate the prerequisites necessary to implement blockchain-based distributed local energy markets. In particular this is,
\begin{itemize}
    \item[a)] forecasting net energy consumption respectively production of private consumers and prosumers one time-step ahead based only on their historical consumption respectively production data (and potentially calender features),
    \item[b)] evaluate and quantify the effects of forecasting errors, i.e., deviations between forecasted and actual consumption respectively production, for households participating in a LEM, and
    \item[c)] evaluate the implications of variations in forecasting quality for a market mechanism including a settlement mechanism (penalty) for forecasting errors.
\end{itemize}

The underlying setting and technical implementation of the local energy market that is assumed for the present research, is provided by \citet{Mengelkamp:2018a}. The prediction task is fitted to their setup of a local energy market that uses blockchain technology as its information and communication medium and thereby, the present study distinguishes itself notably from previous studies solely trying to forecast smart meter time series in general. Likewise, the evaluation of forecasting errors and their implications is based on their described market mechanism and forecasting error settlement structure and has as such to the authors knowledge not been done in other studies. Accordingly, the following research questions are examined in this study:
\begin{itemize}
    \item[a)] Which prediction technique yields the best 15-minute ahead forecast  for smart meter time series measured in 3-minute intervals using only input features generated from the historical values of the time series and calendar-based features?
    \item[b)] Assuming a forecasting error settlement structure as described in \citet{Mengelkamp:2018a}, what is the quantified loss of households participating in the local energy market due to forecasting errors by the prediction technique identified in a)?
    \item[c)] Depending on the results from b), what implications and potential adjustments for the market mechanism described in \citet{Mengelkamp:2018a} can be identified?
\end{itemize}

The remainder of this thesis is structured as follows: Section~\ref{Sec:Method} presents the forecasting models and the error measures used to evaluate their prediction accuracy. Furthermore, it introduces the market mechanism and the implementation of the market simulation which is used to evaluate the effect of prediction errors on market outcomes. Thereafter, Section~\ref{Sec:Data} describes in detail the data used for this study. As the data has not been used in previous studies, emphasis is put on exposing the characteristics and potential peculiarities of the data at hand. Section~\ref{Sec:Results} presents the prediction results of the forecasting models, evaluates their performance relative to a benchmark model and assesses the effect of prediction errors on market outcomes. The insights gained from this will then be used to identify implications and potential adjustments for future market mechanisms that could be implemented as smart contract in a blockchain. Finally, Section~\ref{Sec:Conc} concludes with a summary, limitations of this study, and an outlook on further research questions emerging from the findings of this thesis.

%%%%%%%%%%%%%%%%%%%%%%%%%%%%%%%%%%%%%%%%%%%%%%%%%%%%%%%%%%%%%%%%%