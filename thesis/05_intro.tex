
\section{Introduction}\label{Sec:Intro}



%%%%%%%%%%%%%%%%%%%%%%
%%%   Motivation   %%%
%%%%%%%%%%%%%%%%%%%%%%

\subsection{Motivation}\label{Sec:Intro;Subsec:Motivation}

\begin{itemize}
    \item average single household energy consumption per year: 2300 kWh \\
    $\xrightarrow{}$ 2300/365/24/20 = 0.013 kWh/3minutes on average
    \item fridge: between 60 and 500 kWh per year\\
    $\xrightarrow{}$ 0.0068 - 0.0571 kWh/h
\end{itemize}

The increasingly wide-spread installation of renewable energy generators currently trans-forms the German energy landscape substantially (Bayer et al. 2018). Already in 2016, al-most 1.6 million photovoltaic micro-generation units were installed in Germany, according to the Bundesverband Solarwirtschaft (BSW-Solar 2017). This increasing amount of distrib-uted renewable energy resources combined with a more volatile energy consumption of households – e.g., due to uncontrolled electric vehicle charging that can increase peak con-sumption (Fitzgerald et al. 2016; Le Floch 2017) – presents a serious challenge for grid op-erators. As in an electricity grid energy production and consumption always have to be bal-anced (Weron 2006), the increasingly volatile and hard to predict energy consumption and production in low voltage grids requires new technological solution to manage grid load and energy distribution. \footnote{test}
Fortunately, the technological advancement, that lead to the increasing complexity in the energy landscape, also opens up new opportunities to increase the efficiency and reliability of distributed renewable energy production and distribution. As the amount of renewable energy production, that is fed into low voltage grids, has been increasing over the last years (Bayer et al. 2018), it seems reasonable to shift part of the grid management to lower grid levels. While industry and research already established a comprehensive set of grid manage-ment solutions as well as sophisticated consumption and production forecasting techniques for highly aggregated levels, there is still little research on the same topics at lower aggrega-tion levels, such as neighbourhoods or even individual households (van der Meer et al. 2018).
One rather recent technological advancement that has the potential to increase the level of energy distribution efficiency on low aggregation level is the implementation of local energy markets on a distributed ledger technology, such as Blockchain. Blockchain has been called an invention similarly revolutionary and paradigm shifting as the Internet (Swan 2015). While much of the hype around Blockchain still has to stand the test against reality, the technology undeniably has the potential to enable new technological solutions. It is not for no reason that more than 20 % of 70 surveyed German energy executives believe Block-chain will be “a game changer for the energy industry” and further 60 % believe further dispersion of Blockchain technology is probable (Burger et al. 2016). A use case that has been getting special attention due to the media-effective inauguration of the Brooklyn Mi-crogrid (Rutkin 2016) is Blockchain-based local energy markets. Local energy markets (LEM) enable localized interconnected energy consumers, producers, and prosumers to trade locally produced energy on a market platform with a specific pricing mechanism (Mengelkamp et al. 2018a). Major advantages of such local energy markets are (near) real-time pricing and balancing of energy production and consumption in local grids (ibid.). Blockchain-based LEM utilise a Blockchain as underlying information and communication technology and a Smart Contract to match supply and demand and settle transactions (ibid.).
However, the product traded on energy markets has some peculiarities compared to other markets. First, energy grids always have to be balanced, i.e. energy demand always has to be matched by energy supply (Weron 2006). Secondly, as energy is difficult to store, pro-duced energy is fed into the grid mostly instantaneously and continuously and cannot be exchanged in batches of a specific amount at a single point in time. Traditionally, this means that the aggregated energy demand for a geographic area and a specific period of time has to be anticipated and according to this future demand energy is bought and sold. The actual electricity is then produced continuously matching the current demand. This set-ting is the reasons for today’s existing energy landscape, where utilities and large-scale ener-gy producers and consumers are the only agents involved in electricity markets (Weron 2006). They trade energy according to the aggregated demand of many consumers. This aggregation makes forecasting future energy demand with relatively small errors (van der Meer et al. 2018; Wang et al. 2018 (Early access)) and thereby efficient trading possible. Household-level consumers or prosumers, however, do not actively trade but pay their con-sumption or are reimbursed for their infeed of energy into the grid according to pre-set tar-iffs. In LEM, on the contrary, households are the participating market agents. Due to the peculiarities of energy trading mentioned above, the participating households need to fore-cast their energy demand, respectively supply, to be able to submit a buy or sell offer to the market. This forecasting is substantially harder for single households compared with higher aggregation levels (Wang et al. 2018 (Early access)).
Therefore, the research outlined here aims to evaluate the possibility of providing such forecasts with existing forecasting methods and realistically available Smart Meter data. These forecasts are a necessary precondition for LEM and the blockchain-based LEM envi-sioned by Mengelkamp et al. (2018a) in particular.

%%%%%%%%%%%%%%%%%%%%%%%%%%%%
%%%   Related research   %%%
%%%%%%%%%%%%%%%%%%%%%%%%%%%%

\subsection{Related research}\label{Sec:Intro;Subsec:Related}
The topic of concern for the research proposed touches upon two fields of research. The superordinate topic is local energy markets that are implemented on distributed ledger tech-nology. A necessary prerequisite, that has to be fulfilled to be able to implement such Blockchain-based LEM, is the successful forecast of household-level energy consump-tion/production based on Smart Meter recordings. Without this, trading on the market mechanisms described in Block et al. (2008) and implemented in a Smart Contract Mengelkamp et al. (2018a) is not possible.


%%%%%%%%%%%
\subsubsection{Local energy markets}
While substantial work regarding LEM in general has been done (e.g., Lamparter et al. 2010; Li et al. 2015; Mihaylov et al. 2014), there are only few examples of Blockchain-based LEM designs in the existing literature. 
Mengelkamp et al. (2018b) derive seven principles for microgrid energy markets and evaluate the Brooklyn Microgrid according to those principles. According to the authors knowledge, they are the only ones providing a theoretical framework for the design of Blockchain-based LEM and their work may serve as the basis for the future research and implementation of such energy markets.
With a more practical focus, Mengelkamp et al. (2018a) implemented and simulated a local energy market on a private Ethereum-Blockchain  that enables participants to trade local energy production on a decentralized market platform with no need for a central au-thority.
Münsing et al. (2017) similarly elaborate a peer-to-peer energy market concept on a Blockchain but focus on operational grid constraints and a fair payment rendering. In doing so, they present a decentralized optimal power flow model suitable for implementation on a Blockchain.



%%%%%%%%%%%
\subsubsection{Blockchain and smart contracts}



%%%%%%%%%%%
\subsubsection{Local energy markets and blockchain technology}



%%%%%%%%%%%
\subsubsection{Load forecasting of single households}
As mentioned before, the simulations and concepts described in the research above rely on Smart Meters capable of forecasting the expected energy consumption or production of a household for the next trading interval. This forecasting task is not trivial due to the ex-tremely high volatility of individual private energy consumption (Wang et al. 2018 (Early access)). Nevertheless, there are several studies trying to forecast different time horizons of Smart Meter time series.
Arora and Taylor (2016) compute probability density estimates for the electricity consump-tion recorded by individual Smart Meters in half-hourly intervals from 1000 households and SMEs in Ireland over the course of one year. They employ unconditional and conditional kernel density estimators with a decay parameter to generate point and density forecasts for electricity consumption from 30 minutes to one week ahead.
Kong et al. (2018) use a long short-term memory deep learning framework to make one time-step ahead forecasts on the AMPds dataset containing half-hourly recordings of energy and appliance usage measurements of a single household in Canada. They show that the prediction accuracy can be improved substantially by including appliance measurement da-ta.
Contrary to this machine learning approach, Li et al. (2017) use statistical methods to make one time-step ahead forecasts with a sparse autoregressive LASSO model. Using a dataset of 150 consumers from PG\&E with hourly energy consumption recordings for one year, their model captures sparsity in the household’s historical data via LASSO to make a prediction for one household. This prediction is further improved with the historical con-sumption data of one additional household. This household is identified with the help of a covariance statistic test to identify one other household’s data has the best predictive lever-age to improve the original forecast.
On the same dataset as Arora and Taylor (2016), Shi et al. (2017 (Early access)) use a pooling-based deep recurring neural network to make point forecasts of future consumption and achieve substantial mean absolute percentage error (MAPE) reductions compared to ARIMA, recurring neural network, support vector machine, and deep recurring neural net-work approaches.
Even though focusing on the forecast of aggregated energy consumption, the work of Zufferey et al. (2016) shows promising results for forecasting Smart Meter time series with time delay neural networks (TDNN) using mostly historical features of the time series itself. They use a huge dataset of 40.000 small consumers and 400 photovoltaic power generators in Basel, Switzerland with 15-minute interval recordings of energy consumption and produc-tion for one year.
A comprehensive overview on the state of the art of Smart Meter data analytics is pro-vided by Wang et al. (2018 (Early access)). They not only focus on studies researching load forecasting but also provide comprehensive insights into studies regarding Smart Meter data clustering, preprocessing, load analysis and more. Furthermore, they provide a summary of publicly available Smart Meter datasets and open research topics.
Notably, there is a lack of standard regarding which forecasting error measures are reported and what benchmark models are used in Smart Meter data forecasting studies. This is also pointed out by van der Meer et al. (2018) in their review paper on probabilistic consumption and production forecasting. Due to this, different forecasting techniques employed in studies using different datasets with partly differing objectives are not directly comparable. Which forecasting techniques should be used in the research proposed here is therefore not directly inferable from the success in applying different forecasting techniques in previous studies.



%%%%%%%%%%%%%%%%%%%%%%%%%%%%
%%%   Present research   %%%
%%%%%%%%%%%%%%%%%%%%%%%%%%%%
\subsection{Present research}\label{Sec:Intro;Subsec:Present}



%%%%%%%%%%%
\subsubsection{Objective}
The aim of the proposed Master thesis is to investigate the prerequisites necessary to imple-ment Blockchain enabled distributed energy markets. In particular this is,
a)	forecasting net energy consumption/production of private consumers and prosumers one time-step ahead based only on their historical consumption/production data,
b)	evaluate and quantify the effects of forecasting error for participating households, and
c)	evaluate the implications of variations in forecasting quality for a market mechanism including penalties for forecast-actual consumption/production deviations.
The underlying setting and technical implementation of the local energy market, that is assumed for the proposed research, is provided by Mengelkamp et al. (2018a). The predic-tion task is fitted to their setup of a local energy market that uses Blockchain technology as its information and communication medium and thereby, the proposed study distinguishes itself notably from previous studies trying to forecast Smart Meter time series. Likewise, the evaluation of forecasting errors and their implications is based on their described market mechanism and penalty structure and has as such to the authors knowledge not been done in other studies.


%%%%%%%%%%%
\subsubsection{Research questions}
The following research questions are intended to be answered in the proposed Master thesis:
a)	Which prediction technique yields the best 15-minute ahead forecast  for Smart Meter time series measured in 3-minute intervals using only input features generated from the historical values of the time series and calendar-based features?
b)	Assuming a penalty structure as described in Mengelkamp et al. (2018a), what is the quantified loss of households participating in the local energy market due to forecasting errors by the prediction technique identified in a)?
c)	Depending on the results from b), what implications and potential adjustments for the market mechanism described in Mengelkamp et al. (2018a) can be identified?



%%%%%%%%%%%%%%%%%%%%%%%%%%%%%%%%%%%%%%%%%%%%%%%%%%%%%%%%%%%%%%%%%

\begin{itemize}

    \item What is the subject of the study? Describe the
        economic/econometric problem.

    \item What is the purpose of the study (working hypothesis)?

    \item What do we already know about the subject (literature
        review)? Use citations: {\it \citet{Gallant:87} shows that...
        Alternative Forms of the Wald test are considered
        \citep{Breusch&Schmidt:88}.}

    \item What is the innovation of the study?

    \item Provide an overview of your results.


    \item Outline of the paper:\\
        {\it The paper is organized as follows. The next section describes the
        model under investigation. Section \ref{Sec:Data} describes the data set
        and Section \ref{Sec:Results} presents the results. Finally, Section
        \ref{Sec:Conc} concludes.}

    \item The introduction should not be longer than 4 pages.

\end{itemize}
