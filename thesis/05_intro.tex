
\section{Introduction}\label{Sec:Intro}



%%%%%%%%%%%%%%%%%%%%%%
%%%   Motivation   %%%
%%%%%%%%%%%%%%%%%%%%%%

\subsection{Motivation}\label{Sec:Intro;Subsec:Motivation}

The increasingly wide-spread installation of renewable energy generators currently transforms the German energy landscape substantially \citep{Bayer:2018}. Already in 2016, almost 1.6 million photovoltaic micro-generation units were installed in Germany, according to the Bundesverband Solarwirtschaft \citep{BSW-Solar:2017}. This increasing amount of distributed renewable energy resources combined with a more volatile energy consumption of households –- e.g., due to uncontrolled electric vehicle charging that can increase peak consumption \citep{Fitzgerald:2016,Floch:2017} -– presents a serious challenge for grid operators. As in any electricity grid energy production and consumption always have to be balanced \citep{Weron:2006}, the increasingly volatile and hard to predict energy consumption and production in low voltage grids require new technological solution to manage grid load and energy distribution.

Fortunately, the technological advancement, that lead to the increasing complexity in the energy landscape, also opens up new opportunities to increase the efficiency and reliability of distributed renewable energy production and distribution. As the amount of renewable energy production, that is fed into low voltage grids, has been increasing over the last years \citep{Bayer:2018}, it seems reasonable to shift part of the grid management to lower grid levels. While industry and research already established a comprehensive set of grid management solutions as well as sophisticated consumption and production forecasting techniques for highly aggregated levels, there is still little research on the same topics at lower aggregation levels, such as neighbourhoods or even individual households \citep{Meer:2018}.

One rather recent technological advancement that has the potential to increase the level of energy distribution efficiency on low aggregation level is the implementation of local energy markets on a distributed ledger technology such as blockchain. blockchain has been called an invention similarly revolutionary and paradigm shifting as the Internet \citep{Swan:2015}. While much of the hype around blockchain still has to stand the test against reality, the technology undeniably has the potential to enable new technological solutions. It is not for no reason that more than 20~\% of 70 surveyed German energy executives believe blockchain will be “a game changer for the energy industry” and further 60~\% believe further dispersion of blockchain technology is probable \citep{Burger:2016}. A use case that has been getting special attention due to the media-effective inauguration of the Brooklyn Microgrid \citep{newscientist:2016} is blockchain-based local energy markets.

Local energy markets (LEM) enable localized interconnected energy consumers, producers, and prosumers to trade locally produced energy on a market platform with a specific pricing mechanism \citep{Mengelkamp:2018a}. Major advantages of such local energy markets are (near) real-time pricing and balancing of energy production and consumption in local grids (ibid.). blockchain-based LEM utilise a blockchain as underlying information and communication technology and a smart contract to match supply and demand and settle transactions (ibid.).

However, the product traded on energy markets has some peculiarities compared to other goods. First, energy grids always have to be balanced, i.e. energy demand always has to be matched by energy supply \citep{Weron:2006}. Secondly, as energy is difficult to store, produced energy is fed into the grid mostly instantaneously and continuously and cannot be exchanged in batches of a specific amount at a single point in time. Traditionally, this means that the aggregated energy demand for a geographic area and a specific period of time has to be anticipated and according to this future demand energy is bought and sold. The actual electricity is then produced continuously matching the current demand. This setting is the reasons for today’s existing energy landscape, where utilities and large-scale energy producers and consumers are the only agents involved in electricity markets \citep{Weron:2006}. They trade energy according to the aggregated demand of many consumers. This aggregation makes forecasting future energy demand with relatively small errors \citep{Meer:2018, Wang:2018} and thereby efficient trading possible. Household-level consumers or prosumers, however, do not actively trade but pay their consumption or are reimbursed for their infeed of energy into the grid according to preset tariffs. 

In LEMs, on the contrary, households are the participating market agents. Due to the peculiarities of energy trading mentioned above, the participating households need to forecast their energy demand, respectively supply, to be able to submit a buy or sell offer to the market. This forecasting is substantially harder for single households compared to higher aggregation levels \citep{Wang:2018}.

Therefore, the present research aims to evaluate the possibility of providing such forecasts with existing forecasting methods and realistically available smart meter data with reasonable accuracy. These forecasts are a necessary precondition for LEM and, in particular, the blockchain-based LEM envisioned , e.g., by \citet{Mengelkamp:2018a}.

%%%%%%%%%%%%%%%%%%%%%%%%%%%%
%%%   Related research   %%%
%%%%%%%%%%%%%%%%%%%%%%%%%%%%

\subsection{Related research}\label{Sec:Intro;Subsec:Related}
The present work's topic of concern touches upon three fields of research. The first superordinate topic is local energy markets, their market structure, market mechanism, and market outcomes as well as possible advantages and disadvantages. The second superordinate topic is distributed ledger technology and as such blockchain and smart contracts as well as their use cases for different fields. The third topic is energy forecasting, which encompasses energy consumption forecasting and energy production forecasting. Especially the latter has become increasing attention in the light of increasing adoption of renewable energy resources. All this comes together in blockchain-based local energy markets as implemented in the Brooklyn Microgrid and simulated in the work by \citet{Mengelkamp:2018a}. For such blockchain-based LEM, a necessary prerequisite  is the successful forecast of household-level energy consumption/production based on smart meter recordings. Without this, trading on a market mechanisms as described in \citet{Block:2008} and implemented in a smart contract by \citet{Mengelkamp:2018a} will not possible.


%%%%%%%%%%%
\subsubsection{Local energy markets}
substantial work regarding LEM in general has been done \citep[e.g.,][]{Lamparter:2010, Li:2015, Mihaylov:2014}



%%%%%%%%%%%
\subsubsection{blockchain and smart contracts}



%%%%%%%%%%%
\subsubsection{Local energy markets and blockchain technology}
While substantial work regarding ELM in general has been done (see above), there are only few examples of blockchain-based LEM designs in the existing literature. 
\citet{Mengelkamp:2018b} derive seven principles for microgrid energy markets and evaluate the Brooklyn Microgrid according to those principles. According to the authors knowledge, they are the only ones providing a theoretical framework for the design of blockchain-based LEM and their work may serve as the basis for the future research and implementation of such energy markets.
With a more practical focus, \citet{Mengelkamp:2018a} implemented and simulated a local energy market on a private Ethereum-blockchain  that enables participants to trade local energy production on a decentralized market platform with no need for a central authority.
\citet{Münsing:2017} similarly elaborate a peer-to-peer energy market concept on a blockchain but focus on operational grid constraints and a fair payment rendering. In doing so, they present a decentralized optimal power flow model suitable for implementation on a blockchain.


%%%%%%%%%%%
\subsubsection{Load forecasting of single households}
As mentioned before, the simulations and concepts described in the research above rely on smart meters capable of forecasting the expected energy consumption or production of a household for the next trading interval. This forecasting task is not trivial due to the extremely high volatility of individual private energy consumption \citep{Wang:2018}. Nevertheless, there are several studies trying to forecast different time horizons of smart meter time series.

\citet{Arora:2016} compute probability density estimates for the electricity consumption recorded by individual smart meters in halfhourly intervals from 1000 households and SMEs in Ireland over the course of one year. They employ unconditional and conditional kernel density estimators with a decay parameter to generate point and density forecasts for electricity consumption from 30 minutes to one week ahead.

\citet{Kong:2018} use a long short-term memory deep learning framework to make one time-step ahead forecasts on the AMPds dataset containing half-hourly recordings of energy and appliance usage measurements of a single household in Canada. They show that the prediction accuracy can be improved substantially by including appliance measurement data.

Contrary to this machine learning approach, \citet{Li:2017} use statistical methods to make one time-step ahead forecasts with a sparse autoregressive LASSO model. Using a dataset of 150 consumers from PG\&E with hourly energy consumption recordings for one year, their model captures sparsity in the household’s historical data via LASSO to make a prediction for one household. This prediction is further improved with the historical consumption data of one additional household. This household is identified with the help of a covariance statistic test to identify one other household's data that has the best predictive leverage to improve the original forecast.

On the same dataset as \citet{Arora:2016}, \citet{Shi:2017} use a pooling-based deep recurring neural network to make point forecasts of future consumption and achieve substantial mean absolute percentage error (MAPE) reductions compared to ARIMA, recurring neural network, support vector machine, and deep recurring neural network approaches.

Even though focusing on the forecast of aggregated energy consumption, the work of \citet{Zufferey:2017} shows promising results for forecasting smart meter time series with time delay neural networks (TDNN) using mostly historical features of the time series itself. They use a huge dataset of 40.000 small consumers and 400 photovoltaic power generators in Basel, Switzerland with 15-minute interval recordings of energy consumption and production for one year.

A comprehensive overview on the state of the art of smart meter data analytics is provided by \citet{Wang:2018}. The authors do not only focus on studies researching load forecasting but also provide comprehensive insights into studies regarding smart meter data clustering, preprocessing, load analysis and more. Furthermore, they provide a summary of publicly available smart meter datasets and open research topics.

Notably, there is a lack of standard regarding which forecasting error measures are reported and what benchmark models are used in smart meter data forecasting studies. This is also pointed out by \citet{Meer:2018} in their review paper on probabilistic consumption and production forecasting. Due to this, different forecasting techniques employed in studies using different datasets with partly differing objectives are not directly comparable. 

% Which forecasting techniques should be used in the research proposed here is therefore not directly inferable from the success in applying different forecasting techniques in previous studies.



%%%%%%%%%%%%%%%%%%%%%%%%%%%%
%%%   Present research   %%%
%%%%%%%%%%%%%%%%%%%%%%%%%%%%
\subsection{Present research}\label{Sec:Intro;Subsec:Present}



%%%%%%%%%%%
\subsubsection{Objective}
The aim of the proposed Master thesis is to investigate the prerequisites necessary to implement blockchain-based distributed local energy markets. In particular this is,
\begin{itemize}
    \item[a)] forecasting net energy consumption respectively production of private consumers and prosumers one time-step ahead based only on their historical consumption respectively production data (and potentially calender features),
    \item[b)] evaluate and quantify the effects of forecasting errors, i.e., deviations between forecasted and actual consumption respectively production, for households participating in a LEM, and
    \item[c)] evaluate the implications of variations in forecasting quality for a market mechanism including a settlement mechanism (penalty) for forecasting errors.
\end{itemize}

The underlying setting and technical implementation of the local energy market that is assumed for the present research, is provided by \citet{Mengelkamp:2018a}. The prediction task is fitted to their setup of a local energy market that uses blockchain technology as its information and communication medium and thereby, the present study distinguishes itself notably from previous studies solely trying to forecast smart meter time series in general. Likewise, the evaluation of forecasting errors and their implications is based on their described market mechanism and forecasting error settlement structure and has as such to the authors knowledge not been done in other studies.


%%%%%%%%%%%
\subsubsection{Research questions}
Accordingly, the following research questions are intended to be answered in this study:
\begin{itemize}
    \item[a)] Which prediction technique yields the best 15-minute ahead forecast  for smart meter time series measured in 3-minute intervals using only input features generated from the historical values of the time series and calendar-based features?
    \item[b)] Assuming a forecasting error settlement structure as described in \citet{Mengelkamp:2018a}, what is the quantified loss of households participating in the local energy market due to forecasting errors by the prediction technique identified in a)?
    \item[c)] Depending on the results from b), what implications and potential adjustments for the market mechanism described in \citet{Mengelkamp:2018a} can be identified?
\end{itemize}

The remainder of this thesis is structured as follows: Section~\ref{Sec:Method} presents the forecasting models and the error measures used to evaluate their prediction accuracy. Furthermore, it introduces the market mechanism and the implementation of the market simulation which is used to evaluate the effect of prediction errors on market outcomes. Thereafter, Section~\ref{Sec:Data} describes in detail the data used for this study. As the data has not been used in previous studies, emphasis is put on exposing the characteristics and potential peculiarities of the data at hand. Section~\ref{Sec:Results} presents the prediction results of the forecasting models, evaluates their performance relative to a benchmark model and assesses the effect of prediction errors on market outcomes. The insights gained from this will then be used to identify implications and potential adjustments for future market mechanisms that could be implemented as smart contract in a blockchain. Finally, Section~\ref{Sec:Conc} concludes with a summary, limitations of this study, and an outlook on further research questions emerging from the findings of this thesis.

%%%%%%%%%%%%%%%%%%%%%%%%%%%%%%%%%%%%%%%%%%%%%%%%%%%%%%%%%%%%%%%%%