
\section{Conclusion}\label{Sec:Conc}



%%%%%%%%%%%%%%%%%%%
%%%   Summary   %%%
%%%%%%%%%%%%%%%%%%%%

\subsection{Summary of present research}\label{Sec:Conclusion;Subsec:Summary}

The present research aimed to evaluate, first, the prediction accuracy for household energy consumption and production using state-of-the-art forecasting techniques. Second, to assess the effect of prediction errors in a market simulation using the market mechanism implemented by \citet{Mengelkamp:2018a} in a smart contract on a blockchain. And lastly, to infer the implications of the results for the future design of blockchain-based local energy markets.

For this purpose, first, the performance of two forecasting techniques, which were already successfully applied in previous research, was assessed. A LSTM recurring neural network and a LASSO regression model were fitted on 9 months of consumption respectively production data of German households recorded by smart meters in 3-minute intervals. These models were then used to predict energy consumption respectively production in 15-minute resolution one-step ahead for three months. The predictions were evaluated using several error measures and compared to a benchmark model (na\"ive persistence model). The LASSO model yielded the best results with an average MAPE across all consumer data sets of \textapprox17~\% and was subsequently used to make predictions for the succeeding market simulation. As all prediction models failed to produce satisfactory predictions on the production data, the market simulation used only true production values.

Secondly, the market mechanism implemented by \citet{Mengelkamp:2018a} was used to assess the effect of prediction errors on market outcomes in three different supply scenarios. The evaluation revealed that in a balanced supply and demand scenario the settlement cost due to prediction errors almost completely offset savings introduced by the participation in the local energy market. In an undersupply scenario, the cost due to prediction errors even surpassed the savings and made market participation uneconomical. Only in a scenario with substantial oversupply, the savings brought to consumers by the participation in the local energy market compensated the cost of prediction errors completely.

Thus, thirdly, further possible adjustments necessary for future blockchain-based local energy markets to mitigate this finding were discussed. Here, it was found that this problem would be only diminished but not eliminated by more accurate forecasts. Moreover, it seemed unlikely that the performance of prediction models could be greatly improved without including higher data resolution, behavioural variables, and data from smart appliances (as done by \citet{Kong:2018}) -- which still would not compensate for the unpredictability of human behaviour. Implementing blockchain-based LEM only in market setups with oversupply seems impractical and would most probably diminish the advantages of LEM substantially. Therefore, the most promising approach seemed to be measures that address the market design. This mainly includes adjustments to the market mechanism, which can be two-fold: Either shorter trading periods could introduced which in turn reduces the forecasting horizon and therefore prediction errors or the auction mechanism is altered to use actual consumption values to settle transactions.

Overall, the need to take prediction errors into consideration in the design of blockchain-based market mechanism became evident. This is due to the high uncertainty associated with individual households' energy consumption and therefore also net production patterns that limits the feasibility of accurate forecasts substantially.



%%%%%%%%%%%%%%%%%%%%%%
%%%   Discussion   %%%
%%%%%%%%%%%%%%%%%%%%%%

\subsection{Limitations of present research}\label{Sec:Conclusion;Subsec:Discussion}

There are some limitations to point out of the present work. One major point is that data from more smart meters and more context information about the data would have been desirable. Due to data protection legislation no information regarding locality of the households, the household characteristics or the type of power plant prosumer households used could be provided by Discovergy. This made it difficult to judge the suitability of certain data sets for the market simulation and required a detailed analysis of the energy recordings' patterns of every single data set provided. Also the large share of so declared prosumer data sets without any net energy production readings was unfortunate and unexplained. The large scale differences in the production capacities of the remaining prosumers with net energy production readings complicated the analysis of the market simulation further. Additionally, it would have been preferable to have absolute production and consumption data for prosumers instead of the net consumption respectively production. Although, this circumstance reflected data real-world data availability and is something probably every implementation of blockchain-based LEM would have to deal with. This fact, however, highlights the necessity to improve net demand forecasting as has been already pointed out in previous research as well \citep[e.g.,][]{Meer:2018, Hong:2016}.

The prediction performance of the LSTM model was slightly surprising. The author would have expected better results, especially compared to the LASSO regression model. Here, a major constrained for more elaborate model architectures, the inclusion of more data points and more sophisticated and granular hyperparameter tuning was computing resources. The computing resources available were either not optimized for large scale neural network training (i.e., a lack of graphical processing units (GPUs) capable of tensor operations) or prohibitively expensive to use exceeding the free trial credits for computing resources (i.e., the Google Cloud Platform Free Tier). Especially, the predictions on prosumer data sets could have been much better. Here exist dedicated research fields for the forecasting of electricity production by different type of plants. However, this knowledge could not be adequately put to use in the present work as the households' type of production plants was not known or would have had to be inferred from net production patterns with a high degree of uncertainty.

The evaluation of the predictions on production data also suffered from the unavailability of relative error measures due to the frequent occurence of zero values. The usage of MAPE or NRMSE with the plants production capacity as denominator (as suggested by \citet{Hoff:2013}) would have solved this problem, however, would have required knowledge about the maximum capacity which was again not available. Another promising approach to quantify error measures in the presence of sudden unexpected peaks is the ramp score as utilized in wind energy forecasting and developed on the basis of a swinging door algorithm \citep[e.g.,][]{Bianco:2016, Florita:2013}. This may be useful to include in future similar work.

Finally, it is to mention that the market simulation did not account for taxes or fees, especially grid utilization fees which can be a substantial share of the total electricity cost of households. Moreover, the simulation does not take into account compensation costs for blockchain miners that reimburses them for the computational cost the bear. The modeling of this cost and potential distribution schemes among market participants is definitively needed in future research on blockchain-based energy markets \citep{Mengelkamp:2018a}.



%%%%%%%%%%%%%%%%%%%
%%%   Outlook   %%%
%%%%%%%%%%%%%%%%%%%

\subsection{Outlook and further research}\label{Sec:Conclusion;Subsec:Outlook}

Naturally, future research concerned with blockchain-based LEM should take into account the potential cost of prediction errors. This implies a focus on market mechanisms and prediction error settlement structures that do not make participation in the LEM uneconomical. A special focus on this issue has to be put in situations with undersupply of locally produced energy. A further field of research that already is picking up in sophistication and amount is the forecast of individual household energy consumption and production. While the results of this field are still nowhere close to the forecasting accuracy of aggregated consumption forecasting, there is still room for improvement and refinement of existing prediction techniques. Any advancements made in the prediction of individual households' energy patterns also benefits the blockchain-based LEM research as energy forecasts most likely will play a role in their use cases. Furthermore, to the author's knowledge there has been no simulation of blockchain-based LEM with actual consumption and production data conducted. Doing so on a private blockchain with the market mechanism coded in a smart contract should be the next step for the assessment of potential technological and conceptual weaknesses.

Previous research has shown that blockchain technology and smart contracts can play a valuable role in tackling the challenges of a changing energy landscape. This research emphasizes, however, that advancement on this front cannot be made without a holistic approach that takes all components of blockchain-based local energy markets into account. Simply assuming that reasonably accurate energy forecasts for individual households will be available once the technical challenges of implementing a LEM on a blockchain are solved, may steer research into a wrong direction and miss the opportunity to quickly move into the direction of a more sustainable and less carbon-intensive future.


%%%%%%%%%%%%%%%%%%%%%%%%%%%%%%%%%%%%%%%%%%%%%%%%%%%%%%%%%%%%%%%%%