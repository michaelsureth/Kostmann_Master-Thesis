\section*{Abstract}

Local energy markets (LEMs) have been proposed as a solution to the challenges introduced by the current transformation of the energy landscape towards more distributed and volatile energy production from renewable energy sources. Blockchain-based LEMs take the proposed solution one step further and implement the market mechanism of the LEM as a smart contract. This makes a central authority coordinating the LEM obsolete. Recently proposed blockchain-based LEM designs rely on accurate forecasts of individual households' energy consumption and production to trade in the LEM. In a majority of the literature, such accurate forecasts are simply assumed to be given. The present research tests this assumption by evaluating the forecast accuracy achievable with current state-of-the-art energy forecasting techniques for individual households. In a second step, the effect of prediction errors made by the best performing forecasting technique on market outcomes is assessed in three different supply scenarios. The evaluation shows that, although a LASSO regression model is capable of achieving reasonably low forecasting errors, the costly settlement of prediction errors can offset and even surpass the savings brought to consumers by a blockchain-based LEM. This shows that prediction errors can make the participation in LEMs uneconomical for consumers, and thus, has to be taken into consideration in future research on blockchain-based LEMs.

%%%%%%%%%%%%%%%%%%%%%%%%%%%%%%%%%%%%%%%%%%%%%%%%%%%%%%%%%%%%%%%%%