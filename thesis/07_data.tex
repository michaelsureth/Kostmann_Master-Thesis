
\section{Data}\label{Sec:Data}



%%%%%%%%%%%%%%%%%%%%%%%
%%%   Data source   %%%
%%%%%%%%%%%%%%%%%%%%%%%

\subsection{Source}\label{Sec:Data;Subsec:Source}



%%%%%%%%%%%%%%%%%%%%%%
%%%   Obtainment   %%%
%%%%%%%%%%%%%%%%%%%%%%

\subsection{Obtainment}\label{Sec:Data;Subsec:Obtainment}

The data used for the present research was provided by Discovergy GmbH and can be downloaded from www.research.discovergy.com. This was the first project to make use of this data. As it was the first time that Discovergy GmbH provided data in this form for research purposes, there was no suitable process to retrieve data from their internal data storage solutions. For this reason, the author had to provide an API client for the Discovergy REST API to export data from pre-selected meters.

As all Discovergy smart meters sent their measurements in real-time to Discovergy\'s servers for storage, visualization and analysis, Discovergy clients can access their meters and measurements through a web application and the Discovergy REST API. Authorized users can therefore interact with predefined endpoints. These endpoints serve as an application programming interface (API). By providing the credentials for their Discovergy account (https://my.discovergy.com/login), developers can send requests to a specified endpoint URL. The API returns to such a request a data object formatted in JavaScript Object Notation (JSON). For example, a user authenticates herself with her account credentials and requests the endpoint GET /meters at the URL https://api.discovergy.com/public/v1. Then the server returns a JSON object containing all meter IDs the user has access to.

To automate this process a client had to be programmed. This client  had to be able to authenticate the user with the provided account credentials, request the readings for one year in 3-minute intervals of all meters specified in a text file, and export them to a specified path. As the API had restrictions on the maximum time span of readings that could be returned depending on the measurement resolution (i.e., 10 days for 3-minute intervals), the client had to to make 37 request per meter to cover the whole year of 2017 in 10 day periods. The client was developed in Java based on the demo client provided in the Discovergy REST API documentation (https://api.discovergy.com/docs/). The client was then sent to an Discovergy employee who used an administrative account with access to a sufficiently large number of smart meters to retrieve the data sets used. The code for the API client can be found in \ref{App:Code:API}

%%%%%%%%%%%%%%%%%%%%%%%%%%%%
%%%   Data description   %%%
%%%%%%%%%%%%%%%%%%%%%%%%%%%%

\subsection{Description}\label{Sec:Data;Subsec:Description}

The data comes in 200 individual csv-files containing each the meter readings of a single smart meter. The readings are recorded in 3-minute intervals and range from 01.01.2017 00:00:00 to 01.01.2018 00:00:00. This translates into 175,201 observations per smart meter. Each smart meter records  energy consumption,  energy production (in the case of prosumers) and power over all phases installed in the meter together with a timestamp in Unix milliseconds. For this research, only energy consumption and production are relevant. In summary, the data used here are 200 individual data sets each containing two time series (energy consumption and energy production) with 175,201 observations evenly spaced in 3-minute intervalls.

The dataset used in the proposed research is provided by Discovergy GmbH (www.discovergy.com). It contains data from 200 Smart Meter of 100 randomly chosen German consumers and prosumers each. Each of the 200 Smart Meter measurements are evenly spaced in 3-minute intervals with a length of 175.201 observations ranging from 01.01.2017 00:00:00 to 01.01.2018 00:00:00.
Below, a preprocessed and correctly formatted sample of the data for consumer 56 (c056) and prosumer 89 (p089) containing 6 Smart Meter readings is shown.

For each consumer\/prosumer, three time series containing different Smart Meter measure-ments are provided: energy (consumption), power, and energy out (production). The first column time contains the time stamp of the measurement, the columns c056\_energy and p089\_energy contain the Smart Meter reading of energy consumption of the respective con-sumer/prosumer at the time stamp in the unit 10 -10 kWh. The columns c056\_power and p089\_power contain the power consumed at the time stamp in the unit 10 -3 W. The col-umns c056\_energyOut and p089\_energyOut contain the Smart Meter reading of energy production at the time stamp in the unit 10 -10 kWh. As the c056 is a consumer and not a prosumer, all energyOut values are zero. Note however, that also some prosumers’ ener-gyOut values may for all measurement points be zero for various reasons (e.g., their energy production never exceeded their energy consumption, which is why they never fed in any energy into the grid). 


%%%%%%%%%%%
\subsubsection{}



%%%%%%%%%%%%%%%%%%%%%%%%%%%%%%%%%%%%%%%%%%%%%%%%%%%%%%%%%%%%%%%%%

\begin{itemize}

    \item Describe the data and its quality.
    \item How was the data sample selected?
    \item Provide descriptive statistics such as:
        \begin{itemize}
            \item time period,
            \item number of observations, data frequency,
            \item mean, median,
            \item min, max, standard deviation,
            \item skewness, kurtosis, Jarque--Bera statistic,
            \item time series plots, histogram.
        \end{itemize}
    \item For example:
        \begin{table}[ht]

        \begin{center}
            {\footnotesize
            \begin{tabular}{l|cccccccccc}
                \hline \hline
                           & 3m    & 6m    & 1yr   & 2yr   & 3yr   & 5yr   & 7yr   & 10yr  & 12yr  & 15yr   \\
                \hline
                    Mean   & 3.138 & 3.191 & 3.307 & 3.544 & 3.756 & 4.093 & 4.354 & 4.621 & 4.741 & 4.878  \\
                    StD    & 0.915 & 0.919 & 0.935 & 0.910 & 0.876 & 0.825 & 0.803 & 0.776 & 0.768 & 0.762  \\
                \hline \hline
            \end{tabular}}
        \end{center}
        \caption{Some descriptive statistics of location and dispersion for
        2100 observed swap rates for the period from February 15, 1999
        to March 2, 2007. Swap rates measured as 3.12 (instead of 0.0312). See Table
        \ref{Tab:DescripStatsRawDataDetail} in the appendix for
        more details.}
        \label{Tab:DescripStatsRawData}
        \end{table}

    \item Allows the reader to judge whether the sample is biased or to evaluate possible impacts of outliers, for
    example.

\end{itemize}
