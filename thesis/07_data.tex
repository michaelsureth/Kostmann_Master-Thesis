
\section{Data}\label{Sec:Data}



%%%%%%%%%%%%%%%%%%%%%%%
%%%   Data source   %%%
%%%%%%%%%%%%%%%%%%%%%%%

\subsection{Source}\label{Sec:Data;Subsec:Source}



%%%%%%%%%%%%%%%%%%%%%%
%%%   Obtainment   %%%
%%%%%%%%%%%%%%%%%%%%%%

\subsection{Obtainment}\label{Sec:Data;Subsec:Obtainment}

The data used for the present research was provided by Discovergy GmbH and can be downloaded from www.research.discovergy.com. The data comes in 200 individual csv-files containing the meter readings of 200 smart meters installed and maintained by Discovergy GmbH

%%%%%%%%%%%%%%%%%%%%%%%%%%%%
%%%   Data description   %%%
%%%%%%%%%%%%%%%%%%%%%%%%%%%%

\subsection{Description}\label{Sec:Data;Subsec:Description}

The dataset used in the proposed research is provided by Discovergy GmbH (www.discovergy.com). It contains data from 200 Smart Meter of 100 randomly chosen German consumers and prosumers each. Each of the 200 Smart Meter measurements are evenly spaced in 3-minute intervals with a length of 175.201 observations ranging from 01.01.2017 00:00:00 to 01.01.2018 00:00:00.
Below, a preprocessed and correctly formatted sample of the data for consumer 56 (c056) and prosumer 89 (p089) containing 6 Smart Meter readings is shown.

For each consumer/prosumer, three time series containing different Smart Meter measure-ments are provided: energy (consumption), power, and energy out (production). The first column time contains the time stamp of the measurement, the columns c056_energy and p089_energy contain the Smart Meter reading of energy consumption of the respective con-sumer/prosumer at the time stamp in the unit 10 -10 kWh. The columns c056_power and p089_power contain the power consumed at the time stamp in the unit 10 -3 W. The col-umns c056_energyOut and p089_energyOut contain the Smart Meter reading of energy production at the time stamp in the unit 10 -10 kWh. As the c056 is a consumer and not a prosumer, all energyOut values are zero. Note however, that also some prosumers’ ener-gyOut values may for all measurement points be zero for various reasons (e.g., their energy production never exceeded their energy consumption, which is why they never fed in any energy into the grid). 


%%%%%%%%%%%
\subsubsection{}



%%%%%%%%%%%%%%%%%%%%%%%%%%%%%%%%%%%%%%%%%%%%%%%%%%%%%%%%%%%%%%%%%

\begin{itemize}

    \item Describe the data and its quality.
    \item How was the data sample selected?
    \item Provide descriptive statistics such as:
        \begin{itemize}
            \item time period,
            \item number of observations, data frequency,
            \item mean, median,
            \item min, max, standard deviation,
            \item skewness, kurtosis, Jarque--Bera statistic,
            \item time series plots, histogram.
        \end{itemize}
    \item For example:
        \begin{table}[ht]

        \begin{center}
            {\footnotesize
            \begin{tabular}{l|cccccccccc}
                \hline \hline
                           & 3m    & 6m    & 1yr   & 2yr   & 3yr   & 5yr   & 7yr   & 10yr  & 12yr  & 15yr   \\
                \hline
                    Mean   & 3.138 & 3.191 & 3.307 & 3.544 & 3.756 & 4.093 & 4.354 & 4.621 & 4.741 & 4.878  \\
                    StD    & 0.915 & 0.919 & 0.935 & 0.910 & 0.876 & 0.825 & 0.803 & 0.776 & 0.768 & 0.762  \\
                \hline \hline
            \end{tabular}}
        \end{center}
        \caption{Some descriptive statistics of location and dispersion for
        2100 observed swap rates for the period from February 15, 1999
        to March 2, 2007. Swap rates measured as 3.12 (instead of 0.0312). See Table
        \ref{Tab:DescripStatsRawDataDetail} in the appendix for
        more details.}
        \label{Tab:DescripStatsRawData}
        \end{table}

    \item Allows the reader to judge whether the sample is biased or to evaluate possible impacts of outliers, for
    example.

\end{itemize}
