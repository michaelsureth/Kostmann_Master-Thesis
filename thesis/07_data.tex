
\section{Data}\label{Sec:Data}



%%%%%%%%%%%%%%%%%%%%%%%
%%%   Data source   %%%
%%%%%%%%%%%%%%%%%%%%%%%

\subsection{Source}\label{Sec:Data;Subsec:Source}

The data used for the present research was provided by Discovergy GmbH and can be downloaded from www.research.discovergy.com. Discovergy GmbH installs and maintains smart meters in German households for a one-time installation and monthly maintenance fee. Customers in return get various services centered around the analysis and visualization of their energy consumption and/or production. Discovergy describes itself as a full-range supplier of smart metering solutions offering transparent energy consumption and production data for private and commercial clients. All energy measurements of Discovergy smart meters are accessible through a web portal and mobile app. Additionally, various services are offered, such as, tips for energy savings potential, irregular consumption pattern warnings, personal energy reporting, and consumption analysis of individual appliances.

To be able to offer such data-driven services, Discovergy smart meters\footnote{Discovergy currently installs for private household clients the EasyMeter Q3D standard load profile meter which is connected to the Discovergy Meteorit TM smart meter gateway which records and transmits the recordings to Discovergy servers. The meter specifications can be found here: \texttt{https://discovergy.com/files/sources/product-information/SLP\_Zaehler.pdf} (in German).} record energy consumption and production near real-time -- i.e., 2-second intervals –- and send the readings to Discovergy's servers for storage and analysis. Therefore, Discovergy has extremely high resolution energy data of their customers at their disposal. This high resolution is in stark contrast to the half-hourly or even hourly recorded data used in previous studies (e.g., \cite{Arora:2016,Auder:2018,Shi:2017,Gerossier:2017}).

To the authors knowledge, there is no research using Discovergy smart meter data, apart from \cite{Teixeira:2017} who used the data as simulation input but not for analysis or prediction. As it was the first time that Discovergy provided data in this form for research purposes, there was no suitable process to retrieve data from their internal data storage solutions. For this reason, the author had to provide an API client for the Discovergy REST API to export data from pre-selected meters.


%%%%%%%%%%%%%%%%%%%%%%
%%%   Obtainment   %%%
%%%%%%%%%%%%%%%%%%%%%%

\subsection{Obtainment}\label{Sec:Data;Subsec:Obtainment}

As all Discovergy smart meters send their measurements in real-time to Discovergy\'s servers for storage, visualization and analysis, Discovergy clients can access their meters and measurements through a web application. Moreover, authorized users can interact with the stored meter measurements through predefined endpoints. These endpoints serve as an application programming interface (API) called Discovergy REST API. By providing the credentials for their Discovergy account\footnote{Sign up for a Discovergy account is open to everybody at https://my.discovergy.com/login. This provides access to the Discovergy API for developers, without the need of being a Discovergy client. However, only Discovergy clients will have access to smart meters in their account as their installed Discovergy meters are associated to their account.}, developers can send requests to a specified endpoint URL. The API returns to such a request a data object formatted in JavaScript Object Notation (JSON). For example, a user authenticates herself with her account credentials and requests the endpoint \texttt{GET /meters} at the base URL \texttt{https://api.discovergy.com/public/v1}. In response, the server returns a JSON object containing all meter IDs the user has access to.

To automate this process, the author of this study had to program an API client compliant with the constraints of a RESTful architecture.\footnote{REST refers to Representational State Transfer and describes an architetural style that ensures interoperability of systems through the web \citep[see][]{fielding:2000}.} This client  had to be able to authenticate the user with the provided account credentials, request the readings for one year in 3-minute intervals of all meters specified in a text file, and export them to a specified path. As the API had restrictions on the maximum time span of readings that could be returned depending on the measurement resolution (i.e., returns at most 10 days in 3-minute resolution), the client had to to make 37 request per meter to cover the whole year of 2017 in 10 day periods. As mentioned above, the measurement resolution of the Discovergy smart meters is with 2-second intervals much higher than the 3-minute intervals requested. However, the data management system employed by Discovergy already provides 3-minute aggregations of the original recordings which can be retrieved by specifying the according parameter in the API client.

The client was developed in Java based on the demo client provided in the Discovergy REST API documentation (https://api.discovergy.com/docs/). The client was then sent to an Discovergy employee who used an administrative account with access to a sufficiently large number of smart meters to retrieve the data sets used in this study. The code for the API client can be found in Appendix \ref{App:Code:C1API}.

%%%%%%%%%%%%%%%%%%%%%%%%%%%%
%%%   Data description   %%%
%%%%%%%%%%%%%%%%%%%%%%%%%%%%

\subsection{Description}\label{Sec:Data;Subsec:Description}

The data comes in 200 individual csv-files each containing the meter readings of a single smart meter. The readings are recorded in 3-minute intervals and range from 01.01.2017 00:00:00 to 01.01.2018 00:00:00. This translates into 175,201 observations per smart meter. Each smart meter measures energy consumption,  energy production (in the case of prosumers) and power over all phases installed in the meter and records them together with a timestamp in Unix milliseconds. For this research, only energy consumption and production are relevant. In summary, the data used here are 200 individual data sets each containing two time series (energy consumption and energy production) with 175,201 observations evenly spaced in 3-minute intervals.

Below, a preprocessed and correctly formatted sample of the data for consumer 56 and prosumer 89 containing 6 measurement points are shown.

\begin{table}[h]
    \csvreader[centered tabular=c|cc,
    table head=
    \hline\hline
    \textbf{time} & \textbf{energy} & \textbf{energyOut} \\
    \hline
    \ldots & \ldots & \ldots \\,
    head to column names,
    separator=comma,
    respect all,
    late after line=\\,
    table foot=
    \ldots & \ldots & \ldots \\\hline\hline]
    {tables/consumer-00000056_glimpse.csv}{}%
    {\csvcolii & \csvcoliii & \csvcoliv}%
    \caption[Data excerpt consumer 056]{Data excerpt consumer 056. \quantnet}
\end{table}

\begin{table}[h]
    \csvreader[centered tabular=c|cc,
    table head=
    \hline\hline
    \textbf{time} & \textbf{energy} & \textbf{energyOut} \\
    \hline
    \ldots & \ldots & \ldots \\,
    head to column names,
    separator=comma,
    respect all,
    late after line=\\,
    table foot=
    \ldots & \ldots & \ldots \\\hline\hline]
    {tables/producer-00000089_glimpse.csv}{}%
    {\csvcolii & \csvcoliii & \csvcoliv}%
    \caption[Data excerpt of prosumer 089]{Data excerpt consumer 089. \quantnet}
\end{table}

The energy and energy out measurements are recorded in the unit $10^{-10}$ kWh. As consumer 056 is not a prosumer and has no energy production capacity installed, all energy out measurements must be zero. Note however, that although the data excerpt of prosumer 089 shown here has positive energy out values, there may be prosumers with all zero energy out recordings if their production capacity never exceeds their own consumption. In this case the prosumer never actually feeds energy into the grid and the meter records a energy out reading of zero at all measurement points.
For all further computations, the first order differences of the energy consumption and production readings were calculated. These first order differences are equivalent to the energy consumption/production within each 3-minute interval between two meter recordings. The result of this computation leaves each time series with 175,200 observations.\footnote{One regular year (no leap year) comprises 175,200 3-minute intervals: $365\text{d} * 24\text{h/d} * \frac{60\text{m/h}}{3\text{m}} = 175,200$}



%%%%%%%%%%%
\subsubsection{Consumer data sets}

Figure \ref{Fig:energycons_c082} exemplary shows the energy consumption time series of consumer 082. For easier readability, the consumption has been converted from $10^{-10}$ kWh to kWh. As can be seen in the first panel, the consumption per 3-minute interval fluctuates between 0 and 0.361 kWh with a mean of 0.039 kWh and a median of 0.024 kWh.\footnote{For comparison, an average German single household consumes 2300 kWh per year. This is equivalent to 0.013 kWh per 3-minute interval.} Notably, there are two extended (in March and June) and three shorter periods (in July, September, and December) of clearly distinguishable low consumption and low fluctuations levels. The most likely explanations for these low stable energy consumption periods are holidays in which the household members are on vacation. This also emphasizes, that household members' behaviour is the biggest driver in energy consumption fluctuations and uncertainty of the time series. Interestingly, the time series also shows an increase in mean consumption starting with October 2017. This could be explained by colder outside temperatures, however, within the first quarter of 2017 no equivalent decrease in the mean energy consumption can be seen. Therefore, the reason for this increase might be due to newly acquired household appliances which are increasingly used as the household members spend more time indoors with the approaching winter. 
\begin{figure}[h]
 \centering
\includegraphics[width=\textwidth]{thesis/graphs/timeseries/c082_cons.pdf}
\caption[Energy consumption recordings for consumer 082]{Energy consumption recordings for consumer 082. First panel shows full year 2017, second panel zooms in to one month (May), third panel zooms in to one day (May, 13). \quantnet}
\label{Fig:energycons_c082}
\end{figure}
The second panel zooms to just one month making daily fluctuation patterns already visible. In May there seem to be no abnormal consumption patterns. there are a few peaks in the first and third week of May that stand out, but no longer periods of very low energy consumption. More interesting seems to be the last panel which zooms in to just one day of energy consumption, i.e. May 13, 2017. This day was chosen for no particular reason other than that it is more or less in the middle of the month shown in the second panel. May 13, however, nicely exemplifies a usual pattern of energy consumption: There is only low energy consumption from midnight until about 7.30 a.m. which also fluctuates in a systematic and repeated way. Most probably, this "base" consumption is caused by appliances in standby and "always on" appliances, such as a fridge and/or freezer. At around 7.30 a.m. the household members get up and the energy consumption spikes for the next 30 minutes -- the light is turned on, the coffee machine runs, maybe the stove is turned on, and maybe a flow heater is used to shower with hot water. As the household members leave the house (May 13 is a Monday), the consumption slowly decreases again. In the evening at about 6.30 p.m. energy consumption starts to spike again, probably due to dinner preparations (microwave, stove). Not intuitively explainable is the spike which is visible just before midnight. This spike again highlight the extreme uncertainty contained in individual household energy consumption. It is mostly caused by human behavior, which can seem quite erratic by just looking at energy consumption patterns without context.

To get a better impression of the representativity of consumer 082's the average energy consumption and its variability compared to the other data sets available for this study, Figure \ref{}

\begin{figure}[h]
 \centering
\includegraphics[width=\textwidth]{thesis/graphs/consumer_boxplots_energy.pdf}
\caption[xxx]{xxx. \quantnet}
\label{Fig:boxplots_energy}
\end{figure}

%%%%%%%%%%%
\subsubsection{Prosumer data sets}
%%%%%%%%%%%%%%%%%%%%%%%%%%%%%%%%%%%%%%%%%%%%%%%%%%%%%%%%%%%%%%%%%

\begin{itemize}

    \item Describe the data and its quality.
    \item How was the data sample selected?
    \item Provide descriptive statistics such as:
        \begin{itemize}
            \item time period,
            \item number of observations, data frequency,
            \item mean, median,
            \item min, max, standard deviation,
            \item skewness, kurtosis, Jarque--Bera statistic,
            \item time series plots, histogram.
        \end{itemize}
    \item For example:
        \begin{table}[ht]

        \begin{center}
            {\footnotesize
            \begin{tabular}{l|cccccccccc}
                \hline \hline
                           & 3m    & 6m    & 1yr   & 2yr   & 3yr   & 5yr   & 7yr   & 10yr  & 12yr  & 15yr   \\
                \hline
                    Mean   & 3.138 & 3.191 & 3.307 & 3.544 & 3.756 & 4.093 & 4.354 & 4.621 & 4.741 & 4.878  \\
                    StD    & 0.915 & 0.919 & 0.935 & 0.910 & 0.876 & 0.825 & 0.803 & 0.776 & 0.768 & 0.762  \\
                \hline \hline
            \end{tabular}}
        \end{center}
        \caption{Some descriptive statistics of location and dispersion for
        2100 observed swap rates for the period from February 15, 1999
        to March 2, 2007. Swap rates measured as 3.12 (instead of 0.0312). See Table
        \ref{Tab:DescripStatsRawDataDetail} in the appendix for
        more details.}
        \label{Tab:DescripStatsRawData}
        \end{table}

    \item Allows the reader to judge whether the sample is biased or to evaluate possible impacts of outliers, for
    example.

\end{itemize}
